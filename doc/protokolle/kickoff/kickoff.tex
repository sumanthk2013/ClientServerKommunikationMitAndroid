%
%  Kickoff Protokoll
%
%  Created by Roman Wuersch on 2010-12-12.
%
% =============================================================================
% Documentdefinition  beginns here
% =============================================================================

\documentclass[listof=totoc,bibliography=totoc]{scrreprt}
\usepackage[ngerman]{babel}

\usepackage{tocbasic}

% Use utf-8 encoding for foreign characters
\usepackage[utf8]{inputenc}
% \usepackage[applemac]{inputenc}

% Setup for fullpage use
\usepackage{fullpage}

% Running Headers and footers
%\usepackage{fancyhdr}

% Multipart figures
%\usepackage{subfigure}

% More symbols
%\usepackage{amsmath}
%\usepackage{amssymb}
%\usepackage{latexsym}

% Surround parts of graphics with box
\usepackage{boxedminipage}

% Package for including code in the document
\usepackage{listings}

% If you want to generate a toc for each chapter (use with book)
\usepackage{minitoc}

% This is now the recommended way for checking for PDFLaTeX:
\usepackage{ifpdf}

\ifpdf
    \usepackage[pdftex]{graphicx}
\else
    \usepackage{graphicx}
\fi

% =============================================================================
% Documenttext beginns here
% =============================================================================

\title{Kick-off Protokoll}
    
\author{Studierender - Roman Würsch\\
    Projektbetreuer - Beat Seeliger\\
    \\
    HSZ-T - Technische Hochschule Zürich}
    
\date{14. Dezember 2010}

\begin{document}

    \ifpdf
        \DeclareGraphicsExtensions{.pdf, .jpg, .tif}
    \else
        \DeclareGraphicsExtensions{.eps, .jpg}
    \fi

    \maketitle

    \pagenumbering{arabic}

    % \tableofcontents

    \chapter{Kick-off Protokoll}

    \section{Semesterarbeit}
    Client-Server-Kommunikation mit Android

    \section{Beschlüsse}
    \begin{itemize}
        \item Die Aufgabenstellung wurde gemäss dem Email von Herr Stern
            angepasst.
        \item Als nächster Task, soll eine detailierte Anforderungsanalyse für
            den Prototypen durchgeführt werden.
        \item Es werden die Bewertungskriterien des Bachelor Studienganges als
            Bewertungsgrundlage verwendet.
    \end{itemize}
    
\end{document}