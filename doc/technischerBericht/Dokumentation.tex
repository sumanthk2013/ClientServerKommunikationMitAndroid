%
%   Technische Dokumentation für die Semesterarbeit
%
%   Created by Roman Wuersch on 2010-12-23.	
%
% =============================================================================
% Documentdefinition  beginns here
% =============================================================================

\documentclass[listof=totocnumbered, bibliography=totocnumbered]{scrreprt}
% \documentclass[listof=totoc,bibliography=totoc]{scrreprt}
\usepackage[ngerman]{babel}

\usepackage{tocbasic}

% Use utf-8 encoding for foreign characters
\usepackage[utf8]{inputenc}
% \usepackage[applemac]{inputenc}

% Setup for fullpage use
\usepackage{fullpage}

% Running Headers and footers
%\usepackage{fancyhdr}

% Multipart figures
%\usepackage{subfigure}

% More symbols
%\usepackage{amsmath}
%\usepackage{amssymb}
%\usepackage{latexsym}

% Surround parts of graphics with box
\usepackage{boxedminipage}

% Package for including code in the document
\usepackage{listings}

% If you want to generate a toc for each chapter (use with book)
\usepackage{minitoc}

% Abkürzungsverzeichnis erstellen.
\usepackage{acronym}
% \usepackage[printonlyused]{acronym}

% This is now the recommended way for checking for PDFLaTeX:
\usepackage{ifpdf}

\usepackage[hyperfootnotes=false]{hyperref}
\hypersetup{
  bookmarks=true,         % show bookmarks bar?
  unicode=true,           % non-Latin characters in Acrobat’s bookmarks
  pdftoolbar=true,        % show Acrobat’s toolbar?
  pdfmenubar=true,        % show Acrobat’s menu?
  pdffitwindow=true,      % window fit to page when opened
  pdfstartview={FitH},    % fits the width of the page to the window
  pdftitle={Semesterarbeit},   
  pdfauthor={Roman Würsch},
  pdfsubject={Client-Server-Kommunikation mit Android},
  pdfcreator={TeXnicCenter 1.0 RC1},
  pdfproducer={MiKTeX 2.9},
  pdfnewwindow=true,      % links in new window
  colorlinks=false,       % false: boxed links; true: colored links
  linkcolor=red,          % color of internal links
  citecolor=green,        % color of links to bibliography
  filecolor=magenta,      % color of file links
  urlcolor=cyan          % color of external links
}

\ifpdf
    \usepackage[pdftex]{graphicx}
\else
    \usepackage{graphicx}
\fi

\title{Client-Server-Kommunikation mit Android}

\author{Studierender - Roman Würsch\\
	Projektbetreuer - Beat Seeliger\\
	\\
	HSZ-T - Technische Hochschule Zürich}

\date{\today}

% =============================================================================
% Documenttext beginns here
% =============================================================================

\begin{document}

  \ifpdf
    \DeclareGraphicsExtensions{.pdf, .jpg, .tif}
  \else
    \DeclareGraphicsExtensions{.eps, .jpg}
  \fi
  
  % ===========================================================================
  % Titelblatt beginns here
  % ===========================================================================
  
  \maketitle
  
  \pagenumbering{Roman}
  
  % ===========================================================================
  % Inahltsverzeichnis beginns here
  % ===========================================================================
  
  \tableofcontents
  
  % ===========================================================================
  % Kapitel Administratives beginns here
  % ===========================================================================
  
  \chapter{Administratives}
  
  \section{Personalienblatt}
  \begin{tabbing}
    \hspace*{6cm}\= \kill
    Name, Vorname: \> {\bf Roman Würsch} \\
    Adresse: \> {\bf Murhaldenweg 16} \\
    PLZ, Wohnort: \> {\bf 8057 Zürich} \\
    \\
    Geburtsdatum: \> {\bf 10.11.1980} \\
    Heimatort: \> {\bf Emmetten NW} \\
  \end{tabbing}
  Ich bestätige, dass die vorliegende Semesterarbeit
  ``Client-Server-Kommunikation mit Android'' in allen Teilen selbständig
  erarbeitet und durchgeführt wurde.
  \\
  \\
  \\
  \\
  \\
  \begin{tabbing}
    \hspace*{6cm}\= \kill
    Ort und Datum \> {Unterschrift} \\
  \end{tabbing}
  
  \newpage
  
  \section{Aufgabenstellung}
  
  \subsection{Ausgangslage}
  Die App-Stores der Mobil-Hersteller boomen. Die Art und Weise wie man Apps
  programmieren soll wird von den Mobil-Hersteller meist klar vorgegeben.
  Sobald jedoch eine App mit einem Server im Internet kommunizieren soll, gibt
  es keine Vorschriften zur Implementierung.
  
  \subsection{Ziel der Arbeit}
  Es soll eine Client-Server-Kommunikation am Beispiel des Android
  Betriebsystems gezeigt werden. Das Android Betriebsystem nimmt den
  Platz des Clients ein. Als Server wird ein Java EE Applikationsserver
  verwendet.\newline
  
  Folgende Ziele sollen erreicht werden:
  
  \begin{itemize}
    \item Es sollen zwei gängige plattformunabhängigen Arten von
          Daten-Kommunikation miteinander verglichen werden.
    \item Es solle eine detaillierte Anforderungsanalyse an einen
          Prototypen durchgeführt werden.
    \item Es soll ein Prototyp für Daten-Kommunikation implementiert
          werden. Der Prototyp soll beim Server Daten lesen, erstellen,
          aktualisieren und löschen können. Der Prototyp soll auf einem
          Android-Gerät lauffähig sein.
  \end{itemize}
  
  Folgende Punkte werden abgegrenzt, da es den Rahmen der Arbeit sprengen 
  würde:
  
  \begin{itemize}
    \item Der Vergleich der Daten-Kommunikation soll auf zwei Protokolle
          beschränkt werden, die in der Praxis eingesetzt werden,
          \ac{REST}\footnote[1]{
            Der Begriff Representational State Transfer (mit dem Akronym
            REST) bezeichnet einen Softwarearchitekturstil für verteilte
            Hypermedia-Informationssysteme wie das World Wide Web}
          und Hessian\footnote[2]{
            Hessian ist ein binäres Netzwerkprotokoll, mit dessen Hilfe
            Daten zwischen Systemen ausgetauscht und Remote Procedure Calls
            durchgeführt werden können.}
          .
    \item Der Prototyp des Clients wird für die Android Version 2.2 (Froyo)
          entwickelt und soll mit dem Protokoll \ac{REST}\footnotemark[1] und
          Hessian\footnotemark[2] implementiert werden.
    \item Es werden keine Umfragen, Erhebungen und Feldstudien
          durchgeführt.
  \end{itemize}
  
  \newpage
  
  \subsection{Aufgabenstellung}
  
  \begin{itemize}
    \item Gegenüberstellung der beiden Protokollkonzepte
    \item Gegenüberstellung der Einsatzgebiete der beiden Protokolle
    \item Prüfen ob eine Implementierung für Android möglich ist
    \item Testszenarien für die beiden Prototypen ausarbeiten
    \item Entwicklung eines Prototypen für beide Protokolle \ac{REST} und
          Hessian
    \item Durchführen der definierten Tests der beiden Prototypen
  \end{itemize}
  
  \subsection{Erwartete Resultate}
  Die erwarteten Resultate ergeben sich aus der Aufgabenstellung:
  
  \begin{itemize}
    \item Es wird die Versionskontrolle auf der Basis von GIT bei
          Github.com verwendet.
    \item Planung, Arbeitsnachweis und weitere Informationen werden im WIKI
          von Github.com geführt.
    \item Technischer Bericht
    \begin{itemize}
      \item Beschreibung der Ausgangslage
      \item Konzeptionelle Gegenüberstellung von \ac{REST} und Hessian
      \item Gegenüberstellung der Einsatzgebiete von \ac{REST} und Hessian
      \item Definition der Anforderungen an den Prototyp 
      \item Zusammenfassung zur Entwicklung des Prototypen
      \item Ergebnisse der Tests und Abdeckung der Testszenarios
    \end{itemize}
    \item Lauffähiger Prototyp:
    \begin{itemize}
      \item Es soll ein lauffähiger Prototyp auf der Basis von
            Android 2.2 (Froyo) für beide Protokolle gemacht werden.
      \item Der Prototyp soll bei einem Server Daten lesen, erstellen,
            aktualisieren und löschen können.
    \end{itemize}
  \end{itemize}
  
  \section{Termine}
  
  \begin{tabbing}
    \hspace*{4cm}\= \kill
    Kick-Off:               \> 14. Dezember 2010 \\
    Review:                 \> 12. Januar 2011 \\
    Schlusspräsentation:    \> im Februar 2011 (termin noch nicht bekannt)\\
  \end{tabbing}
  
  \chapter{Technischer Bericht}
  
  \pagenumbering{arabic}
  
  \section{Beschreibung der Ausgangslage}
  
  
  \newpage
  
  \section{Konzeptionelle Gegenüberstellung von REST und Hessian}
  
  
  \newpage
  
  \section{Gegenüberstellung der Einsatzgebiete von REST und Hessian}
  
  
  \newpage
  
  \section{Definition der Anforderungen an den Prototyp}
  
  Die Definition der Anforderungen an die beiden Prototypen werden mit der
  Technik der User Stories\cite{UserStories} gemacht.
  
  \subsection{Was sind User Stories}
  
  User Stories beschreiben Anforderungen an eine Software in eine für Jedermann
  verständlichen Sprache. Es sollen keine technischen Details genannt, sondern
  viel eher Eigenschaften erläutert werden die jeder versteht. Oft werden User
  Stories als Aktionen in einem \ac{GUI} verfasst. Aus den User Stories kann
  man die jeweiligen Akzeptanztests\cite{AcceptanceTests} ableiten.
  
  Eine User Story sollte nicht mehr als drei Sätze haben. Das kommt daher, dass
  eine User Story nie zu komplex sein darf. Wenn man mehr als drei Sätz für die
  Beschreibung einer User Story braucht, sollte diese in mehrere kleinere User
  Stories aufgeteilt werden, welche genug simpel sind, um wiederum in drei
  Sätzen beschrieben zu werden.
  
  \subsection{Anforderungen an den Prototyp aus Sicht des Anwenders}
  
  Der Anwender stellt ein Mensch dar, der nur die Sicht auf die
  Clientapplikation hat. Normalerweise hat ein Anwender keinerleit technische
  Hintergründe.
  
  \newcounter{userStoriesZaehler}
  
  \begin{enumerate}
    \item Ich als Anwender lege keinerlei Wert auf das Aussehen eines
    \ac{GUI}'s
    \item Ich als Anwender lege keinerlei Wert auf die Art und Weise der
    Übertragung der Daten.
    \item Ich als Anwender will meine Applikation überall verwenden können.
    \item Ich als Anwender will eine stabile Applikation.
    \item Ich als Anwender will ein ausgewähltes Datenobjekt vom Server an mein
    Androidgerät übertragen und darstellen können.
    \item Ich als Anwender will eine komplette Liste von Datenobjekten vom
    Server an mein Androidgerät übertragen und darstellen können.
    \item Ich als Anwender will ein ausgewähltes Datenobjekt auf dem Server
    löschen können.
    \item Ich als Anwender will ein ausgewähltes Datenobjekt auf dem Server
    editieren können.
    \item Ich als Anwender will ein neues Datenobjekt an den Server übertragen
    können. Dieses Datenobjekt soll für zukünftige Zugriffe auf dem Server
    gespeichert werden.
    \item Ich als Anwender will über Übertragungsfehler informiert werden.
    \setcounter{userStoriesZaehler}{\value{enumi}}
  \end{enumerate}
  
  \subsection{Anforderungen an den Prototyp aus Sicht des Informatikstudenten}
  
  Der Informatikstudent stellt ein Mensch dar, der sowohl die Sicht auf den
  Client, wie auch auf den Server der Applikation hat. Der Informatikstudent
  versucht durch technische Hilfsmittel die Anforderungen eines Anwenders zu
  erfüllen.
  
  \subsubsection{Clientapplikation}
  
  \begin{enumerate}
    \setcounter{enumi}{\value{userStoriesZaehler}}
    \item Ich als Informatikstudenten lege keinerlei Wert auf das Aussehen
    eines \ac{GUI}'s
    \item Ich als Informatikstudenten will, dass für die Übertragung eines
    Datenobjekts möglichst wenig Rohdaten gesendet und empfangen werden.
    \item Ich als Informatikstudenten will, dass die Clientapplikation für
    Geräte mit der Android Version 2.2 lauffähig ist.
    \item Ich als Informatikstudenten will, dass die Clientapplikation
    unabhängig von der Serverapplikation ausgeliefert werden kann.
    \setcounter{userStoriesZaehler}{\value{enumi}}
  \end{enumerate}
  
  \subsubsection{Serverapplikation}
  
  \begin{enumerate}
    \setcounter{enumi}{\value{userStoriesZaehler}}
    \item Ich als Informatikstudenten will, dass die Serverapplikation immer
    verfügbar ist.
    \item Ich als Informatikstudenten will, dass mehrere Clientapplikationen
    gleichzeitig mit der Serverapplikation bedient werden können.
    \item Ich als Informatikstudenten will, dass die Serverapplikation
    auf dem Internet zugänglich ist.
    \item Ich als Informatikstudenten will, dass die Serverapplikation
    unabhängig von der Clientapplikation ausgeliefert werden kann.
    \setcounter{userStoriesZaehler}{\value{enumi}}
  \end{enumerate}
  
  \subsection{Akzeptanztests, abgeleitet von den User Stories}
  
  \newpage
  
  \section{Zusammenfassung zur Entwicklung des Prototypen}
  
  
  \newpage
  
  \section{Ergebnisse der Tests und Abdeckung der Testszenarios}
  
  
  % ===========================================================================
  % Anhang beginns here
  % ===========================================================================
  
  \appendix
  
  \chapter{Ein Anhang}
  
  Im Anhang kann auf Implementierungsaspekte wie Datenbankschemata
  oder Programmcode eingegangen werden.
  
  % ===========================================================================
  % Abkürzungsverzeichnis beginns here
  % ===========================================================================
  
  \chapter{Abkürzungsverzeichnis}
  \begin{acronym}
    \setlength{\itemsep}{-\parsep}
    \acro{GUI}{Graphical User Interface}
    \acro{REST}{Representational State Transfer}
    \acro{XML}{Extensible Markup Language}
    \acro{JSON}{JavaScript Object Notation}
  \end{acronym}
  
  % ===========================================================================
  % Abbildungsverzeichnis beginns here
  % ===========================================================================
  
  % Abbildungsverzeichnis
  \listoffigures
  
  % ===========================================================================
  % Tabellenverzeichnis beginns here
  % ===========================================================================
  
  % Tabellenverzeichnis
  \listoftables
  
  % ===========================================================================
  % Literaturverzeichnis beginns here
  % ===========================================================================
  
  % verwendet alpha
  \bibliographystyle{alpha}
  % verwendet literaturverzeichnis.bib
  % \renewcommand\bibname{Literaturverzeichnis} % Titel überschreiben
  \cleardoublepage
  \bibliography{literaturverzeichnis}

\end{document}