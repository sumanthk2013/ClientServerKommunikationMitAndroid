%
%   Technische Dokumentation für die Semesterarbeit
%
%   Created by Roman Wuersch on 2010-12-23.	
%
% =============================================================================
% Documentdefinition  beginns here
% =============================================================================

\documentclass[listof=totocnumbered, bibliography=totocnumbered]{scrreprt}
% \documentclass[listof=totoc,bibliography=totoc]{scrreprt}
\usepackage[ngerman]{babel}

\usepackage{tocbasic}

% Use utf-8 encoding for foreign characters
\usepackage[utf8]{inputenc}
% \usepackage[applemac]{inputenc}

% Setup for fullpage use
\usepackage{fullpage}

% Running Headers and footers
%\usepackage{fancyhdr}

% Multipart figures
%\usepackage{subfigure}

% More symbols
%\usepackage{amsmath}
%\usepackage{amssymb}
%\usepackage{latexsym}

% Surround parts of graphics with box
\usepackage{boxedminipage}

% Package for including code in the document
\usepackage{listings}

% If you want to generate a toc for each chapter (use with book)
\usepackage{minitoc}

% Abkürzungsverzeichnis erstellen.
\usepackage{acronym}
% \usepackage[printonlyused]{acronym}

% This is now the recommended way for checking for PDFLaTeX:
\usepackage{ifpdf}

\ifpdf
    \usepackage[pdftex]{graphicx}
\else
    \usepackage{graphicx}
\fi

\title{Client-Server-Kommunikation mit Android}

\author{Studierender - Roman Würsch\\
	Projektbetreuer - Beat Seeliger\\
	\\
	HSZ-T - Technische Hochschule Zürich}

\date{27. Dezember 2010}

% =============================================================================
% Documenttext beginns here
% =============================================================================

\begin{document}

  \ifpdf
    \DeclareGraphicsExtensions{.pdf, .jpg, .tif}
  \else
    \DeclareGraphicsExtensions{.eps, .jpg}
  \fi
  
  % ===========================================================================
  % Titelblatt beginns here
  % ===========================================================================
  
  \maketitle
  
  \pagenumbering{Roman}
  
  % ===========================================================================
  % Inahltsverzeichnis beginns here
  % ===========================================================================
  
  \tableofcontents
  
  % ===========================================================================
  % Kapitel Administratives beginns here
  % ===========================================================================
  
  \chapter{Administratives}
  
  \section{Personalienblatt}
  \begin{tabbing}
    \hspace*{6cm}\= \kill
    Name, Vorname: \> {\bf Roman Würsch} \\
    Adresse: \> {\bf Murhaldenweg 16} \\
    PLZ, Wohnort: \> {\bf 8057 Zürich} \\
    \\
    Geburtsdatum: \> {\bf 10.11.1980} \\
    Heimatort: \> {\bf Emmetten NW} \\
  \end{tabbing}
  Ich bestätige, dass die vorliegende Semesterarbeit
  ``Client-Server-Kommunikation mit Android'' in allen Teilen selbständig
  erarbeitet und durchgeführt wurde.
  \\
  \\
  \\
  \\
  \\
  \begin{tabbing}
    \hspace*{6cm}\= \kill
    Ort und Datum \> {Unterschrift} \\
  \end{tabbing}
  
  \newpage
  
  \section{Aufgabenstellung}
  
  \subsection{Ausgangslage}
  Die App-Stores der Mobil-Hersteller boomen. Die Art und Weise wie man Apps
  programmieren soll wird von den Mobil-Hersteller meist klar vorgegeben.
  Sobald jedoch eine App mit einem Server im Internet kommunizieren soll, gibt
  es keine Vorschriften zur Implementierung.
  
  \subsection{Ziel der Arbeit}
  Es soll eine Client-Server-Kommunikation am Beispiel des Android
  Betriebsystems gezeigt werden. Das Android Betriebsystem nimmt den
  Platz des Clients ein. Als Server wird ein Java EE Applikationsserver
  verwendet.\newline
  
  Folgende Ziele sollen erreicht werden:
  
  \begin{itemize}
      \item Es sollen zwei gängige plattformunabhängigen Arten von
          Daten-Kommunikation miteinander verglichen werden.
      \item Es solle eine detaillierte Anforderungsanalyse an einen
          Prototypen durchgeführt werden.
      \item Es soll ein Prototyp für Daten-Kommunikation implementiert
          werden. Der Prototyp soll beim Server Daten lesen, erstellen,
          aktualisieren und löschen können. Der Prototyp soll auf einem
          Android-Gerät lauffähig sein.
  \end{itemize}
  
  Folgende Punkte werden abgegrenzt, da es den Rahmen der Arbeit sprengen 
  würde:
  
  \begin{itemize}
      \item Der Vergleich der Daten-Kommunikation soll auf zwei Protokolle
          beschränkt werden, die in der Praxis eingesetzt werden,
          REST\footnote{
              Der Begriff Representational State Transfer (mit dem Akronym
              REST) bezeichnet einen Softwarearchitekturstil für verteilte
              Hypermedia-Informationssysteme wie das World Wide Web}
          und Hessian\footnote{
              Hessian ist ein binäres Netzwerkprotokoll, mit dessen Hilfe
              Daten zwischen Systemen ausgetauscht und Remote Procedure Calls
              durchgeführt werden können.}
          .
      \item Der Prototyp des Clients wird für die Android Version 2.2 (Froyo)
          entwickelt und soll mit dem Protokoll REST und Hessian
          implementiert werden.
      \item Es werden keine Umfragen, Erhebungen und Feldstudien
          durchgeführt.
  \end{itemize}
  
  \newpage
  
  \subsection{Aufgabenstellung}
  
  \begin{itemize}
      \item Gegenüberstellung der beiden Protokollkonzepte
      \item Gegenüberstellung der Einsatzgebiete der beiden Protokolle
      \item Prüfen ob eine Implementierung für Android möglich ist
      \item Testszenarien für die beiden Prototypen ausarbeiten
      \item Entwicklung eines Prototypen für beide Protokolle REST und
          Hessian
      \item Durchführen der definierten Tests der beiden Prototypen
  \end{itemize}
  
  \subsection{Erwartete Resultate}
  Die erwarteten Resultate ergeben sich aus der Aufgabenstellung:
  
  \begin{itemize}    
      \item Es wird die Versionskontrolle auf der Basis von GIT bei
          Github.com verwendet.
      \item Planung, Arbeitsnachweis und weitere Informationen werden im WIKI
          von Github.com geführt.
      \item Technischer Bericht
      \begin{itemize}
          \item Beschreibung der Ausgangslage
          \item Konzeptionelle Gegenüberstellung von REST und Hessian
          \item Gegenüberstellung der Einsatzgebiete von REST und Hessian
          \item Definition der Anforderungen an den Prototyp 
          \item Zusammenfassung zur Entwicklung des Prototypen
          \item Ergebnisse der Tests und Abdeckung der Testszenarios
          
      \end{itemize}
      \item Lauffähiger Prototyp:
      \begin{itemize}
          \item Es soll ein lauffähiger Prototyp auf der Basis von
              Android 2.2 (Froyo) für beide Protokolle gemacht werden.
           \item Der Prototyp soll bei einem Server Daten lesen, erstellen,
              aktualisieren und löschen können.
      \end{itemize}
  \end{itemize}
  
  \section{Termine}
  
  \begin{tabbing}
      \hspace*{4cm}\= \kill
    Kick-Off:               \> 14. Dezember 2010 \\
    Review:                 \> 12. Januar 2011 \\
    Schlusspräsentation:    \> im Februar 2011 (termin noch nicht bekannt)\\
  \end{tabbing}
  
  \chapter{Inhalt}
  
  \pagenumbering{arabic}
  
  \section{Beschreibung der Ausgangslage}
  
  
  \section{Konzeptionelle Gegenüberstellung von REST und Hessian}
  
  
  \section{Gegenüberstellung der Einsatzgebiete von REST und Hessian}
  
  
  \section{Definition der Anforderungen an den Prototyp}
  
  
  \section{Zusammenfassung zur Entwicklung des Prototypen}
  
  
  \section{Ergebnisse der Tests und Abdeckung der Testszenarios}
  
  
  % ===========================================================================
  % Anhang beginns here
  % ===========================================================================
  
  \appendix
  
  \chapter{Ein Anhang}
  
  Im Anhang kann auf Implementierungsaspekte wie Datenbankschemata
  oder Programmcode eingegangen werden.
  
  % ===========================================================================
  % Abkürzungsverzeichnis beginns here
  % ===========================================================================
  
  \chapter{Abkürzungsverzeichnis}
  \begin{acronym}
    \setlength{\itemsep}{-\parsep}
    \acro{XML}{Extensible Markup Language}
    \acro{JSON}{JavaScript Object Notation}
  \end{acronym}
  
  % ===========================================================================
  % Abbildungsverzeichnis beginns here
  % ===========================================================================
  
  % Abbildungsverzeichnis
  \listoffigures
  
  % ===========================================================================
  % Tabellenverzeichnis beginns here
  % ===========================================================================
  
  % Tabellenverzeichnis
  \listoftables
  
  % ===========================================================================
  % Literaturverzeichnis beginns here
  % ===========================================================================
  
  % verwendet alpha
  \bibliographystyle{alpha}
  % verwendet literaturverzeichnis.bib
  % \renewcommand\bibname{Literaturverzeichnis} % Titel überschreiben
  \cleardoublepage
  \bibliography{literaturverzeichnis}

\end{document}