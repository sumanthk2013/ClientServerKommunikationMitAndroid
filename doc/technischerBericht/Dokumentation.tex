%
%   Technische Dokumentation für die Semesterarbeit
%
%   Created by Roman Wuersch on 2010-12-23.	
%
% =============================================================================
% Documentdefinition  beginns here
% =============================================================================

\documentclass[listof=totocnumbered, bibliography=totocnumbered]{scrreprt}
% \documentclass[listof=totoc,bibliography=totoc]{scrreprt}
\usepackage[ngerman]{babel}

\usepackage{tocbasic}

% Use utf-8 encoding for foreign characters
\usepackage[utf8]{inputenc}
% \usepackage[applemac]{inputenc}

% Setup for fullpage use
\usepackage{fullpage}

% Running Headers and footers
%\usepackage{fancyhdr}

% Multipart figures
%\usepackage{subfigure}

% More symbols
%\usepackage{amsmath}
%\usepackage{amssymb}
%\usepackage{latexsym}

% Surround parts of graphics with box
\usepackage{boxedminipage}

% Package for including code in the document
\usepackage{listings}

% If you want to generate a toc for each chapter (use with book)
\usepackage{minitoc}

% Abkürzungsverzeichnis erstellen.
\usepackage[printonlyused]{acronym}

% schöne Tabelle zeichnen
\usepackage{booktabs}
\renewcommand{\arraystretch}{1.2} %Die Zeilenabstände in Tabllen angepasst.

% für variable Breiten
\usepackage{tabularx}

% Durchgestrichener Text
\usepackage[normalem]{ulem} %emphasize weiterhin kursiv

% This is now the recommended way for checking for PDFLaTeX:
\usepackage{ifpdf}

\usepackage[hyperfootnotes=false]{hyperref}
\hypersetup{
  bookmarks=true,         % show bookmarks bar?
  unicode=true,           % non-Latin characters in Acrobat’s bookmarks
  pdftoolbar=true,        % show Acrobat’s toolbar?
  pdfmenubar=true,        % show Acrobat’s menu?
  pdffitwindow=true,      % window fit to page when opened
  pdfstartview={FitH},    % fits the width of the page to the window
  pdftitle={Semesterarbeit},   
  pdfauthor={Roman Würsch},
  pdfsubject={Client-Server-Kommunikation mit Android},
  pdfcreator={TeXnicCenter 1.0 RC1},
  pdfproducer={MiKTeX 2.9},
  pdfnewwindow=true,      % links in new window
  colorlinks=false,       % false: boxed links; true: colored links
  linkcolor=red,          % color of internal links
  citecolor=green,        % color of links to bibliography
  filecolor=magenta,      % color of file links
  urlcolor=cyan          % color of external links
}

\ifpdf
    \usepackage[pdftex]{graphicx}
\else
    \usepackage{graphicx}
\fi

\title{Client-Server-Kommunikation mit Android}

\author{Studierender - Roman Würsch\\
	Projektbetreuer - Beat Seeliger\\
	\\
	HSZ-T - Technische Hochschule Zürich}

\date{\today}

% =============================================================================
% Documenttext beginns here
% =============================================================================

\begin{document}

  \ifpdf
    \DeclareGraphicsExtensions{.pdf, .jpg, .tif}
  \else
    \DeclareGraphicsExtensions{.eps, .jpg}
  \fi
  
  % ===========================================================================
  % Titelblatt beginns here
  % ===========================================================================
  
  \maketitle
  
  \pagenumbering{Roman}
  
  % ===========================================================================
  % Inahltsverzeichnis beginns here
  % ===========================================================================
  
  \tableofcontents
  
  % ===========================================================================
  % Kapitel Administratives beginns here
  % ===========================================================================
  
  \chapter{Administratives}
  
  \section{Personalienblatt}
  \begin{tabbing}
    \hspace*{6cm}\= \kill
    Name, Vorname: \> {\bf Roman Würsch} \\
    Adresse: \> {\bf Murhaldenweg 16} \\
    PLZ, Wohnort: \> {\bf 8057 Zürich} \\
    \\
    Geburtsdatum: \> {\bf 10.11.1980} \\
    Heimatort: \> {\bf Emmetten NW} \\
  \end{tabbing}
  Ich bestätige, dass die vorliegende Semesterarbeit
  ``Client-Server-Kommunikation mit Android'' in allen Teilen selbständig
  erarbeitet und durchgeführt wurde.
  \\
  \\
  \\
  \\
  \\
  \begin{tabbing}
    \hspace*{6cm}\= \kill
    Ort und Datum \> {Unterschrift} \\
  \end{tabbing}
  
  \newpage
  
  \section{Bewertungskriterien}
  
  Wie im Kick-off Meeting festgelegt werden die Bewertungskriterien aus dem
  Bachelor Studiengang verwendet. 
  
  \section{Sprache}
  
  Der Bericht wird in deutscher Sprache verfasst. Englische Ausdrücke werden im
  Context verwendet, wenn man davon ausgehen kann, dass es Ausdrücke aus dem
  Gebiet der Informatik sind, die man verstehen sollte.
  
  \section{Termine}
  
  \begin{tabular}{lp{10cm}ll}
    \toprule
    Termin & Datum & Ort \\
    \midrule
    14. 12. 2010 & Kick-off Meeting & Panter Inc\\
    12. 01. 2011 & Design-Review Meeting & HSZ-T\\
    02. 03. 2011 & Schlusspräsentation & HSZ-T\\
    \bottomrule
  \end{tabular}
  
  \section{Projekthistorie}
  
  \begin{tabular}{lp{10cm}ll}
    \toprule
    Datum & Status & Wer \\
    \midrule
    06. 12. 2010 & Ein Dozierender hat die Arbeit inkl. Aufgabenstellung
    ausgeschrieben und wartet auf einen Studierenden der diese Arbeit
    durchführt & Roman Würsch\\
    07. 12. 2010 & Die Arbeit ist freigegeben & Olaf Stern\\
    07. 12. 2010 & Der Kickoff-Termin wurde reserviert & Roman Würsch\\
    14. 12. 2010 & Freigabe Kick-Off & Beat Seeliger\\
    12. 01. 2011 & Freigabe Desgin-Review & Beat Seeliger\\
    \bottomrule
  \end{tabular}
  
  \newpage
  
  \section{Aufgabenstellung}
  
  \subsection{Ausgangslage}
  Die App-Stores der Mobil-Hersteller boomen. Die Art und Weise wie man Apps
  programmieren soll wird von den Mobil-Hersteller meist klar vorgegeben.
  Sobald jedoch eine App mit einem Server im Internet kommunizieren soll, gibt
  es keine Vorschriften zur Implementierung.
  
  \subsection{Ziel der Arbeit}
  Es soll eine Client-Server-Kommunikation am Beispiel des Android
  Betriebsystems gezeigt werden. Das Android Betriebsystem nimmt den
  Platz des Clients ein. Als Server wird ein Java EE Applikationsserver
  verwendet.\newline
  
  Folgende Ziele sollen erreicht werden:
  
  \begin{itemize}
    \item Es sollen zwei gängige plattformunabhängigen Arten von
          Daten-Kommunikation miteinander verglichen werden.
    \item Es solle eine detaillierte Anforderungsanalyse an einen
          Prototypen durchgeführt werden.
    \item Es soll ein Prototyp für Daten-Kommunikation implementiert
          werden. Der Prototyp soll beim Server Daten lesen, erstellen,
          aktualisieren und löschen können. Der Prototyp soll auf einem
          Android-Gerät lauffähig sein.
  \end{itemize}
  
  Folgende Punkte werden abgegrenzt, da es den Rahmen der Arbeit sprengen 
  würde:
  
  \begin{itemize}
    \item Der Vergleich der Daten-Kommunikation soll auf zwei Protokolle
          beschränkt werden, die in der Praxis eingesetzt werden,
          \ac{REST}\footnote[1]{
            Der Begriff Representational State Transfer (mit dem Akronym
            REST) bezeichnet einen Softwarearchitekturstil für verteilte
            Hypermedia-Informationssysteme wie das World Wide Web} und
          Hessian\footnote[2]{
            Hessian ist ein binäres Netzwerkprotokoll, mit
            dessen Hilfe Daten zwischen Systemen ausgetauscht und Remote
            Procedure Calls durchgeführt werden können.}.
    \item Der Prototyp des Clients wird für die Android Version 2.2
          (Froyo\footnote[3]{
            Die Android Entwickler geben den verschiedenen Versionen
            ihres Betriebsystems jeweils Namen von süssen Speisen FroYo
            ist ein Akronym für Frozen yogurt}) entwickelt und soll mit dem
          Protokoll \ac{REST}\footnotemark[1] und Hessian\footnotemark[2]
          implementiert werden.
    \item Es werden keine Umfragen, Erhebungen und Feldstudien
          durchgeführt.
  \end{itemize}
  
  \newpage
  
  \subsection{Aufgabenstellung}
  
  \begin{itemize}
    \item Gegenüberstellung der beiden Protokollkonzepte
    \item Gegenüberstellung der Einsatzgebiete der beiden Protokolle
    \item Prüfen ob eine Implementierung für Android möglich ist
    \item Testszenarien für die beiden Prototypen ausarbeiten
    \item Entwicklung eines Prototypen für beide Protokolle \ac{REST} und
          Hessian
    \item Durchführen der definierten Tests der beiden Prototypen
  \end{itemize}
  
  \subsection{Erwartete Resultate}
  Die erwarteten Resultate ergeben sich aus der Aufgabenstellung:
  
  \begin{itemize}
    \item Es wird die Versionskontrolle auf der Basis von GIT bei
          Github.com verwendet.
    \item Planung, Arbeitsnachweis und weitere Informationen werden im WIKI
          von Github.com geführt.
    \item Technischer Bericht
    \begin{itemize}
      \item Beschreibung der Ausgangslage
      \item Konzeptionelle Gegenüberstellung von \ac{REST} und Hessian
      \item \sout{Gegenüberstellung der Einsatzgebiete von \ac{REST} und
      Hessian}\footnote[4]{
        Im Design-Review hat man sich auf den Verzicht dieses Kapitels
        geeinigt.}
      \item Definition der Anforderungen an den Prototyp 
      \item Zusammenfassung zur Entwicklung des Prototypen
      \item Ergebnisse der Tests und Abdeckung der Testszenarios
    \end{itemize}
    \item Lauffähiger Prototyp:
    \begin{itemize}
      \item Es soll ein lauffähiger Prototyp auf der Basis von
            Android 2.2 (Froyo) für beide Protokolle gemacht werden.
      \item Der Prototyp soll bei einem Server Daten lesen, erstellen,
            aktualisieren und löschen können.
    \end{itemize}
  \end{itemize}
  
  \chapter{Technischer Bericht}
  
  \pagenumbering{arabic}
  
  \section{Beschreibung der Ausgangslage}
  
  Für das Informatik Diplomstudium an der Fachhochschule Zürich für Technik
  HSZ-T wird von den Studenten verlangt eine Semesterarbeit eigenständig zu
  verfassen.
  
  \subsection{Wahl des Themas}
  
  Seit dem Siegeszug der Smartphones, welcher durch das iPhone von Apple
  eingeläutet wurde. 
  
  \newpage
  
  \section{Konzeptionelle Gegenüberstellung von REST und Hessian}
  
  
  \newpage
  
  \section{Definition der Anforderungen an den Prototyp}
  
  Die Definition der Anforderungen an die Prototypen werden mit der Technik der
  User Sto\-ries\cite{UserStories} gemacht.
  
  \subsection{Was sind User Stories}
  
  User Stories beschreiben Anforderungen an eine Software in einer für Jedermann
  verständlichen Sprache. Es sollen keine technischen Details genannt, sondern
  viel eher Eigenschaften erläutert werden die jeder versteht. Oft werden User
  Stories als Aktionen in einem \ac{GUI} verfasst. Aus den User Stories kann
  man die jeweiligen Akzeptanztests\cite{AcceptanceTests} ableiten.
  
  Eine User Story sollte nicht mehr als drei Sätze haben. Das kommt daher, dass
  eine User Story nie zu komplex sein darf. Wenn man mehr als drei Sätz für die
  Beschreibung einer User Story braucht, sollte diese in mehrere kleinere User
  Stories aufgeteilt werden, welche genug simpel sind, um wiederum in drei
  Sätzen beschrieben zu werden.
  
  Jeder User Story wird mit einer eindeutigen Nummer versehen: US-\{User Story
  Nummer\}
  
  \subsection{Anforderungen an den Prototyp}
  
  Im folgenden werden die Anforderungen an den Prototypen spezifziert. Die
  Anforderungen werden aus zwei Blickwinkeln definiert. Aus der Sicht eines
  Anwenders und aus der Sicht eines Informatikstudenten. Diese beiden
  Sichtweisen werde jeweils ich vertreten, da ich mich davon abgegrenzt habe,
  irgenwelche Erhebungen durchzuführen.
  
  Nachfolgend ist die Rede von Datenobjekten. Bei einem Datenobjekt wie es hier
  aufgeführt wird, handelt es sich um ein \ac{POJO}. 
  
  \subsection{Aus der Sicht des Anwenders}
  
  Der Anwender stellt ein Mensch dar, der nur die Sicht auf die
  Clientapplikation hat. Normalerweise hat ein Anwender keinerleit technische
  Hintergründe.\\
  \\
  \begin{tabular}{cp{13cm}l}
    \toprule
    User Story & Beschreibung \\
    \midrule
    US-1 & Ich als Anwender will ein hochwertiges \ac{GUI}. \\
    US-2 & Ich als Anwender will meine Applikation überall verwenden können. \\
    US-3 & Ich als Anwender will eine stabile Applikation. \\
    US-4 & Ich als Anwender will ein ausgewähltes Datenobjekt vom Server an mein
    Androidgerät übertragen und darstellen können. \\
    US-5 & Ich als Anwender will eine komplette Liste von Datenobjekten vom
    Server an mein Androidgerät übertragen und darstellen können. \\
    US-6 & Ich als Anwender will ein ausgewähltes Datenobjekt auf dem Server
    löschen können. \\
    US-7 & Ich als Anwender will ein ausgewähltes Datenobjekt auf dem Server
    editieren können. \\
    US-8 & Ich als Anwender will ein neues Datenobjekt an den Server übertragen
    können. Dieses Datenobjekt soll für zukünftige Zugriffe auf dem Server
    gespeichert werden. \\
    US-9 & Ich als Anwender will über Übertragungsfehler informiert werden. \\
    \bottomrule
  \end{tabular}
  
  \subsection{Aus der Sicht des Informatikstudenten}
  
  Der Informatikstudent stellt ein Mensch dar, der sowohl die Sicht auf den
  Client, wie auch auf den Server der Applikation hat. Der Informatikstudent
  versucht durch technische Hilfsmittel die Anforderungen eines Anwenders zu
  erfüllen.
  
  \subsubsection{Clientapplikation}
  
  \begin{tabular}{cp{13cm}l}
    \toprule
    User Story & Beschreibung \\
    \midrule
    US-10 & Ich als Informatikstudenten will, dass für die Übertragung eines
    Datenobjekts möglichst wenig Rohdaten gesendet und empfangen werden. \\
    US-11 & Ich als Informatikstudenten will, dass die Clientapplikation für
    Geräte mit der Android Version 2.2 lauffähig ist. \\
    US-12 & Ich als Informatikstudenten will, dass die Clientapplikation
    unabhängig von der Serverapplikation ausgeliefert werden kann. \\
    \bottomrule
  \end{tabular}
  
  \subsubsection{Serverapplikation}
  
  \begin{tabular}{cp{13cm}l}
    \toprule
    User Story & Beschreibung \\
    \midrule
    US-13 & Ich als Informatikstudent will, dass die Serverapplikation auf einem
    GlassFish Applikationsserver läuft. \\
    US-14 & Ich als Informatikstudent will, dass die Serverapplikation sowohl
    Requests über das \ac{REST} Protokoll wie auch über das Hessian
    Protokoll bearbeiten kann. \\
    US-15 & Ich als Informatikstudent will, dass Datenobjekte auf der
    Serverapplikation gespeichert werden können. Diese Daten sollen nur solange
    gespeichert sein, wie der Server läuft. \\
    US-16 & Ich als Informatikstudent will, dass mehrere Clientapplikationen
    gleichzeitig mit der Serverapplikation bedient werden können. \\
    US-17 & Ich als Informatikstudent will, dass die Serverapplikation
    auf dem Internet zugänglich ist. \\
    US-18 & Ich als Informatikstudent will, dass die Serverapplikation immer
    verfügbar ist. \\
    US-19 & Ich als Informatikstudent will, dass die Serverapplikation
    unabhängig von der Clientapplikation ausgeliefert werden kann. \\
    \bottomrule
  \end{tabular}
  
  \subsection{Priorisierung aller User Stories}
  
  Da es sich bei dieser Arbeit und eine Semesterarbeit handelt, kann nicht auf
  jede Anforderung eingegangen werden. Die User Stories werden aus meiner Sicht
  priorisiert. Für jede Story, die in dieser Semesterarbeit nicht umgesetzt
  wird, werde ich eine kurze Begründung dazu abgeben.
  
  \subsubsection{Aus der Sicht des Anwenders}
  
  \begin{tabular}{ccc}
    \toprule
    User Story & Umsetzung & Priorisierung \\
    \midrule
    US-1 & Nein & --- \\
    US-2 & Ja & mittel \\
    US-3 & Ja & tief \\
    US-4 & Ja & hoch \\
    US-5 & Ja & hoch \\
    US-6 & Ja & hoch \\
    US-7 & Ja & hoch \\
    US-8 & Ja & hoch \\
    US-9 & Ja & tief \\
    \bottomrule
  \end{tabular}
  
  \subsubsection{Aus der Sicht des Informatikstudenten}
  
  \begin{tabular}{ccc}
    \toprule
    User Story & Umsetzung & Priorisierung \\
    \midrule
    US-10 & Ja & mittel \\
    US-11 & Ja & mittel \\
    US-12 & Ja & tief \\
    US-13 & Ja & mittel \\
    US-14 & Ja & hoch \\
    US-15 & Ja & hoch \\
    US-16 & Ja & tief \\
    US-17 & Nein & --- \\
    US-18 & Nein & --- \\
    US-19 & Ja & tief \\
    \bottomrule
  \end{tabular}
  
  \subsection{Stories die nicht umgesetzt werden}
  
  \begin{itemize}
    \item User Story US-1: Es wurde auf die Umsetzung eines hochwertigen
    \ac{GUI}'s für den Androidclient verzichtet, da es den Aufwand einer
    Semesterarbeit übersteigt.
    \item User Story US-17: Der Server soll nicht über das Internet zugänglich
    sein, da eine lauffähige Instanz eines GlassFish Servers, der von überall
    her im Internet zugänglich ist, Geld kosten würde. Ich bin nicht bereit für
    eine Semesterarbeit einen solchen Hosting Vertrag abzuschliessen.
    \item User Story US-18: Der Server kann nicht immer verfügbar sein, da die
    Instanz des GlassFish Servers auf meinen Notebook laufen wird.
  \end{itemize}
  
  \subsection{Planung der User Stories die umgesetzt werden}
  
  Es werden alle User Stories mit der Priorität hoch und mittel umgesetzt. Die
  Reihenfolge der Umsetztung wird von mir festgelegt. Ich versuche die
  Reihenfolge so festzulegen, damit es ein Sinnvoller Ablauf in der Entwicklung
  der einzelnen User Stories gibt.
  
  Falls die Zeit reicht, werden auch noch die tief priorisierten Users Stories
  umgesetzt. Es wird keine Reihenfolge für tief priorisierte User Stories
  festgelegt. Die Reihenfolge der Umsetzung darf somit frei gewählt werden.\newline
  
  Es wird folgende Reihenfolge für die Umsetzung definiert:

  \subsubsection{Priorität hoch}
  
  \newcounter{userStoriesZaehler}
  
  \begin{enumerate}
    \item User Story US-15
    \item User Story US-14
    \item User Story US-4
    \item User Story US-5
    \item User Story US-8
    \item User Story US-6
    \item User Story US-7
    \setcounter{userStoriesZaehler}{\value{enumi}}
  \end{enumerate}
  
  \subsubsection{Priorität mittel}
  
  \begin{enumerate}
    \setcounter{enumi}{\value{userStoriesZaehler}}
    \item User Story US-13
    \item User Story US-11
    \item User Story US-10
    \item User Story US-2
  \end{enumerate}
  
  \subsection{Testszenario}
  
  Aus den priorisierten User Stories können nun Akzeptanztests abgeleitet
  werden. Eine Userstory ist dann fertig, wenn sie technisch umgesetzt wurde und
  die dazu definierten Akzeptanztests abgenommen wurden. Jedem
  Akzeptanztest wird ein Typ zugewiesen. Der Typ definiert, ob es sich
  dabei um einen Unit-Test [U] oder einen manuell auszuführenden Test [M]
  handelt. Jeder Akzeptanztest wird mit einer eindeutigen Nummer versehen: T-\{User Story\}.\{Testnummer\}\\
  \\
  \begin{tabular}{cccp{11.3cm}l}
    \toprule
    TN & US & Typ & Beschreibung \\
    \midrule
    T-15.1 & US-15 & M & Es soll ein Datenobjekt auf dem Server gespeichert
    werden. Die Referenz des Datenobjekts soll man sich merken.\\
    T-15.2 & US-15 & M & Es soll ein Datenobjekt auf dem Server über
    die Referenz gelesen werden. Es soll die Gleichheit der beiden Datenobjekte
    geprüft werden.\\
    T-14.1 & US-14 & M & Es soll ein \ac{REST} Request auf dem Server abgesetzt
    und auf ein erfolgreiche Rückmeldung geprüft werden.\\
    T-14.2 & US-14 & U & Es soll ein Hessian Request auf dem Server abgesetzt
    und auf ein erfolgreiche Rückmeldung geprüft werden.\\
    T-4.1 & US-4 & M & Es soll ein \ac{REST} GET Request mit der Referenz auf
    eine bekannte Ressource von Prototypen aus auf dem Server abgesetzt und auf
    ein erfolgreiche Rückmeldung geprüft werden.\\
    T-4.2 & US-4 & U & Es soll ein Hessian GET Request mit der Referenz auf
    eine bekannte Ressource auf dem Server abgesetzt und auf ein erfolgreiche
    Rückmeldung geprüft werden.\\
    T-5.1 & US-5 & M & Es soll ein \ac{REST} GET Request auf die
    Listenansicht einer bekannte Ressource von Prototypen aus auf dem Server
    abgesetzt und auf ein erfolgreiche Rückmeldung geprüft werden.\\
    T-5.2 & US-5 & U & Es soll ein Hessian allProducts Request mit der auf dem
    Server abgesetzt und auf ein erfolgreiche Rückmeldung geprüft werden.\\
    \bottomrule
  \end{tabular}
  
  \newpage
  
  \section{Zusammenfassung zur Entwicklung des Prototypen}
  
  
  \newpage
  
  \section{Ergebnisse der Tests und Abdeckung des Testszenarios}
  
  
  % ===========================================================================
  % Anhang beginns here
  % ===========================================================================
  
  \appendix
  
  \chapter{Ein Testanhang}
  
  Im Anhang kann auf Implementierungsaspekte wie Datenbankschemata
  oder Programmcode eingegangen werden.
  
  % ===========================================================================
  % Abkürzungsverzeichnis beginns here
  % ===========================================================================
  
  \chapter{Abkürzungsverzeichnis}
  \begin{acronym}
    \setlength{\itemsep}{-\parsep}
    \acro{GUI}{Graphical User Interface}
    \acro{REST}{Representational State Transfer}
    \acro{POJO}{Plain Old Java Object}
    \acro{XML}{Extensible Markup Language}
    \acro{JSON}{JavaScript Object Notation}
  \end{acronym}
  
  % ===========================================================================
  % Abbildungsverzeichnis beginns here
  % ===========================================================================
  
  % Abbildungsverzeichnis
  \listoffigures
  
  % ===========================================================================
  % Tabellenverzeichnis beginns here
  % ===========================================================================
  
  % Tabellenverzeichnis
  \listoftables
  
  % ===========================================================================
  % Literaturverzeichnis beginns here
  % ===========================================================================
  
  % verwendet alpha
  \bibliographystyle{alpha}
  % verwendet literaturverzeichnis.bib
  % \renewcommand\bibname{Literaturverzeichnis} % Titel überschreiben
  \cleardoublepage
  \bibliography{literaturverzeichnis}

\end{document}