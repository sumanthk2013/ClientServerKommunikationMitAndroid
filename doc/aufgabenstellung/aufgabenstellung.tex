%
%  Antrag, Aufgabenstellung Semesterarbeit
%
%  Created by Roman Wuersch on 2010-12-04.
%
\documentclass[]{scrreprt}
\usepackage[ngerman]{babel}

% Use utf-8 encoding for foreign characters
\usepackage[applemac]{inputenc}

% Setup for fullpage use
\usepackage{fullpage}

% Running Headers and footers
%\usepackage{fancyhdr}

% Multipart figures
%\usepackage{subfigure}

% More symbols
%\usepackage{amsmath}
%\usepackage{amssymb}
%\usepackage{latexsym}

% Surround parts of graphics with box
\usepackage{boxedminipage}

% Package for including code in the document
\usepackage{listings}

% If you want to generate a toc for each chapter (use with book)
\usepackage{minitoc}

% This is now the recommended way for checking for PDFLaTeX:
\usepackage{ifpdf}

\ifpdf
    \usepackage[pdftex]{graphicx}
\else
    \usepackage{graphicx}
\fi

\title{Aufgabenstellung\\
    Semesterarbeit in Informatik}
    
\author{Studierender - Roman W�rsch\\
    Projektbetreuer - Beat Seeliger\\
    \\
    HSZ-T - Technische Hochschule Z�rich}
    
\date{6. Dezember 2010}

\begin{document}

    \ifpdf
        \DeclareGraphicsExtensions{.pdf, .jpg, .tif}
    \else
        \DeclareGraphicsExtensions{.eps, .jpg}
    \fi

    \maketitle

    \pagenumbering{arabic}

    % \tableofcontents

    \chapter{Aufgabenstellung Semesterarbeit}

    \section{Thema}
    Client-Server-Kommunikation mit Android
    
    \section{Ausgangslage}
    Die App-Stores der Mobil-Hersteller boomen. Die Art und Weise wie man Apps
    programmieren soll wird von den Mobil-Hersteller meist klar vorgegeben.
    Sobald jedoch eine App mit einem Server im Internet kommunizieren soll,
    gibt es keine Vorschriften zur Implementierung.

    \section{Ziel der Arbeit}
    Es soll eine Client-Server-Kommunikation am Beispiel des Android
    Betriebsystems gezeigt werden. Das Android Betriebsystem nimmt den
    Platz des Clients ein. Als Server wird ein Java EE Applikationsserver verwendet.\newline
    
    Folgende Ziele sollen erreicht werden:
    
    \begin{itemize}
        \item Es sollen zwei g�ngige plattformunabh�ngigen Arten von Daten-Kommunikation
            miteinander verglichen werden.
        \item Es soll ein Prototyp f�r Daten-Kommunikation implementiert werden.
            Der Prototyp soll beim Server Daten lesen, erstellen, aktualisieren und l�schen
            k�nnen. Der Prototyp soll auf einem Android-Ger�t lauff�hig sein.
    \end{itemize}
    
    Folgende Punkte werden abgegrenzt, da es den Rahmen der Arbeit sprengen 
    w�rde:
    
    \begin{itemize}
        \item Der Vergleich der Daten-Kommunikation soll auf zwei Protokolle beschr�nkt
            werden, die in der Praxis eingesetzt werden, REST\footnote{
                Der Begriff Representational State Transfer (mit dem Akronym REST) bezeichnet
                einen Softwarearchitekturstil f�r verteilte Hypermedia-Informationssysteme wie
                das World Wide Web}
            und Hessian\footnote{
                Hessian ist ein bin�res Netzwerkprotokoll, mit dessen Hilfe Daten zwischen
                Systemen ausgetauscht und Remote Procedure Calls durchgef�hrt werden k�nnen.}
            .
        \item Der Prototyp des Clients wird f�r die Android Version 2.2 (Froyo) entwickelt
            und soll mit dem Protokoll REST und Hessian implementiert werden.
        \item Es werden keine Umfragen, Erhebungen und Feldstudien durchgef�hrt.
    \end{itemize}

    \newpage
    
    \section{Aufgabenstellung}
    
    \begin{itemize}
        \item Gegen�berstellung der beiden Protokollkonzepte
        \item Gegen�berstellung der Einsatzgebiete der beiden Protokolle
        \item Pr�fen ob eine Implementierung f�r Android m�glich ist
        \item Testszenarien f�r die beiden Prototypen ausarbeiten
        \item Entwicklung eines Prototypen f�r beide Protokolle REST und Hessian
        \item Durchf�hren der definierten Tests der beiden Prototypen
    \end{itemize}

    \section{Erwartete Resultate}
    Die erwarteten Resultate ergeben sich aus der Aufgabenstellung:

    \begin{itemize}    
        \item Es wird die Versionskontrolle auf der Basis von GIT bei Github.com verwendet.
        \item Planung, Arbeitsnachweis und weitere Informationen werden im WIKI von
            Github.com gef�hrt.
        \item Technischer Bericht
        \begin{itemize}
            \item Beschreibung der Ausgangslage
            \item Konzeptionelle Gegen�berstellung von REST und Hessian
            \item Gegen�berstellung der Einsatzgebiete von REST und Hessian
            \item Definition der Anforderungen an den Prototyp 
            \item Zusammenfassung zur Entwicklung des Prototypen
            \item Ergebnisse der Tests und Abdeckung der Testszenarios
            
        \end{itemize}
    	\item Lauff�higer Prototyp:
    	\begin{itemize}
            \item Es soll ein lauff�higer Prototyp auf der Basis von Android 2.2 (Froyo)
                f�r beide Protokolle gemacht werden.
             \item Der Prototyp soll bei einem Server Daten lesen, erstellen, aktualisieren und l�schen
                k�nnen.
        \end{itemize}
    \end{itemize}

    \section{Geplante Termine}
    Die Termine k�nnen zum Zeitpunkt des Antrages noch nicht definitiv 
    festgelegt werden. Sofern jedoch die Planung eingehalten werden kann und 
    freie Termine zur Verf�gung stehen, sollten die Termine innerhalb der 
    angegebenen Monate liegen.

    \begin{tabbing}
        \hspace*{4cm}\= \kill
    	Kick-Off:               \> Dezember 2010 \\
    	Review:                 \> Januar 2011 \\
    	Schlusspr�sentation:    \> Februar 2011 \\
    \end{tabbing}

    \section{Genehmigung}
    Der Studierende, sein Projektbetreuer und der Studiengangsleiter 
    Informatik erkl�ren sich mit der Aufgabenstellung einverstanden und geben 
    die Arbeit frei zur Erfassung im Einschreibesystem der Hochschule f�r 
    Technik Z�rich.

    \begin{tabbing}
        \hspace*{10cm}\= \kill
    	Roman W�rsch, Studierender \> Beat Seeliger, Projektbetreuer \\\\\\
        \line(1,0){150} \> \line(1,0){150} \\\\\\
    	Dr. Olaf Stern, Studiengangsleiter Informatik \\\\\\
        \line(1,0){150}
    \end{tabbing}    
\end{document}